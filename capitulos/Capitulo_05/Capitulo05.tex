\documentclass[12pt]{article}
\usepackage[T1]{fontenc}
\usepackage[utf8]{inputenc}
\usepackage[a4paper,top=3.5cm,left=3cm,right=3cm,bottom=2.5cm]{geometry}
\usepackage[brazil]{babel}
\usepackage{graphicx}
\usepackage{hyperref}
\usepackage{fancyhdr}
\usepackage{background}
\usepackage[a4paper,top=3.5cm,left=3cm,right=3cm,bottom=2.5cm]{geometry}
\usepackage{lmodern}
\usepackage{tikz}
\usepackage[font={small,stretch=0.80,it}]{caption}

%Configurando o mapa mental
\usetikzlibrary{mindmap}

%Configurando a path das imagens
\graphicspath{{../../imagens/capitulo5/}}

%Configurando a imagem de background
\backgroundsetup{
scale=1,
angle=0,
opacity=0.4,
contents={%
  \includegraphics[width=\paperwidth,height=\paperheight]{wallpaper.png}
  }%
}

%configurando os hyperlinks
\hypersetup{
    colorlinks=true,
    linkcolor=green,
    filecolor=magenta,      
    urlcolor=blue,
}

%configurando os headers
\pagestyle{fancy}
\fancyhf{}
\rhead{LDO}
\lhead{Capítulo 1}
\rfoot{Página \thepage}

%configurando identação e separação de parágrafos
\parindent 1.27cm
\parskip   6pt

%títulos,autor e data
\title{\textbf{Capítulo 5 \\ Machine Learning e Sistemas Embarcados}}
\author{Gustavo Lopes Rodrigues}
\date{Novembro de 2020}

\begin{document}
    
    %Inserindo o título
    \maketitle

    \begin{center}
        \begin{tikzpicture}[mindmap, grow cyclic, every node/.style=concept, concept color=orange!40,
        level 1/.append style={level distance=5cm,sibling angle=90},
        level 2/.append style={level distance=3cm,sibling angle=45}]

        \node{Novidades em Machine Learning}
            child [concept color=blue!30] { node {GPT-3 \\ \ref{sec:gpt-3}}
                child { node {\href{https://www.sas.com/pt_br/insights/analytics/deep-learning.html}{Deep Learning}}}
                child { node {\href{https://openai.com/}{Open AI}}}
                child { node {\href{https://debuild.co/}{debuild.co}}}
            }
            child [concept color=yellow!30] { node {YouTube \\ \ref{sec:youtube}}
                child { node {\href{https://rockcontent.com/br/blog/algoritmo-do-youtube/}{Algoritmo de pesquisa}}}
                child { node {\href{https://youtu.be/oWcVTWLgACM}{Tags}}}
                child { node {\href{https://super.abril.com.br/tecnologia/como-funciona-a-recomendacao-de-videos-do-youtube/}{Recomen dação}}}
            }
            child [concept color=teal!40] { node {Amazon \\ \ref{sec:amazon}}
                child { node {\href{https://www.noticiastecnologia.com.br/amazon-ajusta-algoritmo-de-pesquisa-para-priorizar-os-proprios-produtos}{Relevância ou Lucro?}}}
                child { node {\href{https://www.visualcapitalist.com/amazon-worlds-most-valuable-retailer/}{Maior varejista dos EUA}}}
            }
            child [concept color=purple!50] { node {Deep Blue \\ \ref{sec:deep_blue}}
                child { node {\href{https://pt.wikipedia.org/wiki/Garry_Kasparov}{Garry Kasparov}}}
                child { node {\href{https://pt.wikipedia.org/wiki/IBM}{IBM}}}
                child { node {\href{https://canaltech.com.br/produtos/o-que-e-supercomputador/}{Super computador}}}
            };
        \end{tikzpicture}

    \end{center}

    %Configurando uma imagem
    \newpage    

    \newpage
    \section*{\centering Material extra}\label{sec:extra} %Criando uma tag para que possa ser referência em outras partes do programa

    %Iniciando listagem
    \begin{itemize}
        \item \href{http://datascienceacademy.com.br/blog/17-casos-de-uso-de-machine-learning/}{\textbf{Machine Learning no cotidiano}}
        \item \href{https://deepmind.com/research/case-studies/alphago-the-story-so-far}{\textbf{Alpha GO}} \\ Obs: \\ \href{https://youtu.be/WXuK6gekU1Y}{\textbf{Também de uma olhada no documentário sobre a Alpha GO}}
        \item \href{https://youtu.be/uGYJuOyIvzs}{\textbf{Aplicação de Machine Learning na saúde}}
        \item \href{https://youtu.be/AwmvwTopbas}{\textbf{Usando Machine Learning para fazer ampliação de vídeos}}
        \item \href{https://forbes.com.br/forbes-insider/2020/07/por-que-o-programa-de-inteligencia-artificial-gpt-3-e-incrivel-mas-superestimado/}{\textbf{Leia mais sobre a GPT-3!}}
    \end{itemize}

    \newpage

    %Iniciando referências
    \begin{thebibliography}{4}
        \bibitem{exemplos} 
        Equipe Interop \\
        \href{https://www.interop.com.br/blog/exemplos-de-machine-learning/}{\textbf{Exemplos de machine Learning}} 
        
        \bibitem{deschamps} 
        Filipe Deschamps \\
        \href{https://www.interop.com.br/blog/exemplos-de-machine-learning/}{\textbf{Agora Aquela I.A. Foi Longe Demais (e vai mudar o jeito que você trabalha)}}

        \bibitem{deepblue} 
        Deep Blue \\
        \href{https://pt.wikipedia.org/wiki/Deep_Blue}{\textbf{Artigo da Wikipedia sobre a Deep Blue}}
        
        \bibitem{deepblue} 
        Garry Kasparov X Deep Blue \\
        \href{https://pt.wikipedia.org/wiki/Deep_Blue}{\textbf{A história das partidas entre Garry Kasparov e Deep Blue}}
    \end{thebibliography}

    

\end{document}