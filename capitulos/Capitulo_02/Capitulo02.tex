\documentclass{article}
\usepackage[T1]{fontenc}
\usepackage[utf8]{inputenc}
\usepackage[a4paper,top=3.5cm,left=3cm,right=3cm,bottom=2.5cm]{geometry}
\usepackage[brazil]{babel}
\usepackage{graphicx}
\usepackage{hyperref}
\usepackage{fancyhdr}
\usepackage{background}
\usepackage[a4paper,top=3.5cm,left=3cm,right=3cm,bottom=2.5cm]{geometry}
\usepackage{lmodern}

%Configurando a path das imagens
\graphicspath{{../../imagens/capitulo2/}}

%Configurando a imagem de background
\backgroundsetup{
scale=1,
angle=0,
opacity=0.4,
contents={%
  \includegraphics[width=\paperwidth,height=\paperheight]{wallpaper.png}
  }%
}

%configurando os hyperlinks
\hypersetup{
    colorlinks=true,
    linkcolor=blue,
    filecolor=magenta,      
    urlcolor=cyan,
}

%configurando os headers
\pagestyle{fancy}
\fancyhf{}
\rhead{LDO}
\lhead{Capítulo 1}
\rfoot{Página \thepage}

%configurando identação e separação de parágrafos
\parindent 1.27cm
\parskip   6pt

\flushbottom

%títulos,autor e data
\title{\textbf{Capítulo 2 \\ Novidades de Machine Learning na atualidade}}
\author{Gustavo Lopes Rodrigues}
\date{Novembro de 2020}

\begin{document}
    
    %Inserindo o título
    \maketitle

    %Aumentando o tamanho padrão das fontes
    \Large

    %Configurando uma imagem
    \begin{figure}[htp]
        \centering
        \includegraphics[scale=0.3]{maxresdefault.jpg}
        \caption{\href{https://youtu.be/pbVwH8o837A}{Agora Aquela I.A. Foi Longe Demais (e vai mudar o jeito que você trabalha)}}
    \end{figure}
    
    %Iniciando a seção 'Inicio'
    \section{Início}

    Eai? Você viu o vídeo acima do \href{https://br.linkedin.com/in/filipedeschamps}{Filipe Deschamps}? 
    Se não eu recomendo fortemente que assista, mas vamos comentar brevemente sobre o assunto 
    do mesmo: a GPT-3.

    Em Junho de 2020, a instituição de pesquisa em Inteligência Artificial,
    \href{https://openai.com/}{Open AI}, lançou as versões iniciais do Generative Pre-trained Transformer 3 também conhecido 
    pelo acrônimo GPT-3, que é um modelo de linguagem autoregressiva que usa 
    aprendizado profundo (\href{https://www.sas.com/pt_br/insights/analytics/deep-learning.html}
    {Deep Learning}) para produzir textos semelhantes aos humanos.

    Na primeira parte do vídeo, o primeiro exemplo dado foi do site \href{https://debuild.co/}{debuild.co} 
    que permite criar aplicativos web, rápidos apenas descrevendo as funcionalidades do programa. 
    Porém, como é demonstrado posteriormente pelo YouTuber, a GPT-3 possui capacidades ainda mais 
    impressionantes, como: criar gráficos, criar planilhas ou até gerar textos de 
    figuras famosas, como \href{https://pt.wikipedia.org/wiki/Elon_Musk}{Elon Musk}.

    Isso é só um exemplo em como Machine Learning ainda é uma área que só tem 
    crescido desde o \href{run:../Capitulo1.pdf}{Game of Checkers}, mas não é 
    o único exemplo em como essa área da computação atua no nosso dia-a-dia.

    Vamos listar aqui alguns exemplos, mas saiba que nas referências desse 
    documento, você encontrará mais links sobre o assunto “Machine Learning 
    no cotidiano”, na seção \ref{sec:extra}

    %Forçar o programa ir para uma nova página
    \newpage
    \section{YouTube} 

    \begin{figure}[htp]
        \centering
        \includegraphics{youtube.png}
    \end{figure}

    Todos os vídeos na plataforma possuem uma característica muito importante
    que são as palavras-chave (também conhecida como ‘tags’). Quando você 
    assiste algum por um bom tempo, o algoritmo do YouTube irá induzir que 
    o usuário gosta do tipo de conteúdo que está no vídeo. Tendo isso em mente,
    ele irá pegar as palavras-chave do tal vídeo, e na próxima vez que o usuário 
    abrir a página de recomendação, os vídeos que estarão na lista serão 
    justamente aqueles que possuem as palavras-chave que o usuário demonstrou interesse.

    \newpage
    \section{Amazon}

    \begin{figure}[htp]
        \centering
        \includegraphics[scale=0.3]{amazon.jpg}
    \end{figure}

    Assim como o YouTube, a Amazon também utiliza de um sistema de recomendações, 
    para direcionar produtos possivelmente relevantes a um cliente, mas além 
    disso, o sistema conta com um algoritmo que a partir de dados de pesquisas
    anteriores, o mesmo aprende o que é mais importante para os clientes na 
    hora de pesquisar determinado produto. Segundo estudos, quando tais 
    sistemas são colocados, \href{https://papers.ssrn.com/sol3/papers.cfm?abstract_id=2263983}
    {eles têm a capacidade de aumentar as compras em até 30\%}


    \newpage
    \section{Conclusão}
    O restante dos exemplos, vamos deixar para a sua curiosidade. Mas acredito que 
    vocês tenham captado a ideia: Machine learning está em um contínuo uso dentro da 
    nossa sociedade e tem sido utilizado para melhorar a experiência dos usuários na internet, 
    direcionando as suas atenções para o conteúdo que são mais atrativos.


    \newpage
    \section*{\centering Material extra}\label{sec:extra} %Criando uma tag para que possa ser referência em outras partes do programa

    %Iniciando listagem
    \begin{itemize}
        \item \href{http://datascienceacademy.com.br/blog/17-casos-de-uso-de-machine-learning/}{\textbf{Machine Learning no cotidiano}}
        \item \href{https://deepmind.com/research/case-studies/alphago-the-story-so-far}{\textbf{Alpha GO}} \\ Obs: \\ \href{https://youtu.be/WXuK6gekU1Y}{\textbf{Também de uma olhada no documentário sobre a Alpha GO}}
        \item \href{https://youtu.be/uGYJuOyIvzs}{\textbf{Aplicação de Machine Learning na saúde}}
        \item \href{https://youtu.be/AwmvwTopbas}{\textbf{Usando Machine Learning para fazer ampliação de vídeos}}
        \item \href{https://forbes.com.br/forbes-insider/2020/07/por-que-o-programa-de-inteligencia-artificial-gpt-3-e-incrivel-mas-superestimado/}{\textbf{Leia mais sobre a GPT-3!}}
    \end{itemize}

    \newpage

    %Iniciando referências
    \begin{thebibliography}{2}
        \bibitem{exemplos} 
        Equipe Interop \\
        \href{https://www.interop.com.br/blog/exemplos-de-machine-learning/}{\textbf{Exemplos de machine Learning}} 
        
        \bibitem{deschamps} 
        Filipe Deschamps \\
        \href{https://www.interop.com.br/blog/exemplos-de-machine-learning/}{\textbf{Agora Aquela I.A. Foi Longe Demais (e vai mudar o jeito que você trabalha)}} 
    \end{thebibliography}

    

\end{document}